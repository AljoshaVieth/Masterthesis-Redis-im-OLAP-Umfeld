\chapter{Implementierung}
Für diese Arbeit wurde der \acl{SSB} sowohl in PostgreSQL, als auch in abgewandelter Form in Redis durchgeführt.
\section{\acl{SSB} in PostgreSQL}
\subsection{Generierung der \acs{SSB}-Daten}
% Hier ganze Anleitung im Anhang verlinken
Die Daten für den \acl{SSB} können mit dem Programm ssb-dbgen~\cite{phillips_electrumssb-dbgen_2023} generiert werden.
Das Programm kann verschiedene Skalierungen der Daten generieren.
Dabei werden .tbl Dateien generiert, die in eine SQL-basierende Datenbank importiert werden können.
\subsection{Ausführen des \acl{SSB} in PostgreSQL}
% Hier ganze Anleitung im Anhang verlinken

Für das Ausführen des \ac{SSB} in PostgreSQL existiert ein GitHub-Repository ~\cite{nukoyokohama_ssb-postgres_2023}~(japanisch) mit einer Anleitung und Skripten.


\section{Arbeitsumfeld}

\subsection{Docker}

% Quelle: Docker-Doku

\section{Praktische Implementierung}

\subsection{Scala}

\subsection{Scala mit Lua}

\section{Limitierungen der Implementierung}
