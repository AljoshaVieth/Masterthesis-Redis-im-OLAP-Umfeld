\chapter{Einleitung}


% TODO: Überprüfen ob Kapitel immer auf rechten Seiten anfangen.
% TODO: Quellen
\section{Hintergrund und Kontext}
Die vorliegende Arbeit untersucht die Eignung von Redis, einer weit verbreiteten Software, die in der Regel als Datenbank, Cache, Message Broker oder Streaming Engine verwendet wird, für Online-Analytical-Processing (OLAP)-Aufgaben. Redis zeichnet sich besonders durch seine Geschwindigkeit und einfache Abfragemöglichkeiten von Daten aus und findet vor allem im Bereich von Online-Transaction-Processing-(OLTP) Anwendung. Der Einsatz von Redis im OLAP-Bereich ist jedoch weitgehend unerforscht, mit einigen wenigen Ausnahmen, in denen Redis lediglich als Cache für vorberechnete Daten verwendet wird.

\section{Forschungsfrage und Zielsetzung}
Diese Arbeit untersucht, ob und inwiefern Redis als eigenständige Lösung für OLAP-Aufgaben genutzt werden kann, ohne dass dabei Daten vorberechnet werden müssen. Der Star Schema Benchmark (SSB), der ursprünglich für relationale Datenbanken entwickelt wurde, wird zur Evaluierung herangezogen. Neben der grundsätzlichen Eignung von Redis für OLAP werden spezifische Aspekte wie Geschwindigkeit, Skalierbarkeit und Anpassungsaufwand von Daten analysiert.

\section{Methodik}
Um die Forschungsfrage zu beantworten, wird untersucht, wie Redis verwendet werden kann, um die Abfragen des SSB Benchmarks auszuführen. Der Benchmark wird üblicherweise zur Messung der Leistung von relationalen Datenbanksystemen eingesetzt, deren Daten im Sternschema vorliegen, das oft in OLAP verwendet wird.
Anschließend erfolgt eine Analyse der Ergebnisse anhand verschiedener Kriterien wie Ausführungsgeschwindigkeit, Skalierbarkeit und Anpassungsaufwand der Datenstruktur.


\section{Gliederung der Arbeit}
Die Arbeit wird im zweiten Kapitel mit einem Überblick über verwandte Forschungsarbeiten eingeleitet.
Im dritten Kapitel werden die technischen Grundlagen erläutert, die für das weitere Verständnis notwendig sind.
Das vierte Kapitel zeigt, wie der \acf{SSB} in PostgreSQL ausgeführt wird.
In Kapitel 5 werden verschiedene Ansätze zur Umsetzung des \ac{SSB} in Redis beschrieben, ausgewertet und miteinander verglichen.
Schließlich wird in Kapitel 6 ein Fazit gezogen und ein Ausblick auf mögliche zukünftige Entwicklungen gegeben.


\section{Einsatz von Künstlicher Intelligenz}
Zum Schreiben dieser Arbeit wurde Künstliche Intelligenz (KI) eingesetzt.
Dabei wurden die Tools \enquote{ChatGPT} (GPT-4)~\cite{openai_chatgpt_nodate} der Firma \emph{OpenAI} sowie \enquote{DeepL Write}~\cite{deepl_se_deepl_nodate} von \emph{DeepL SE} verwendet.

\emph{ChatGPT} wurde genutzt, um aus den eigens verfassten Sätzen oder Stichpunkten des Autors ausführlichere Texte zu generieren.
Zusätzlich kam \emph{DeepL Write} zum Einsatz, um die Texte weiter zu verfeinern und ihnen einen akademischen Schreibstil zu verleihen.

KI wurde nicht verwendet, um selbstständig komplette Texte zu formulieren oder neue Informationen zu generieren.
Stattdessen wurden ausschließlich die vom Autor bereitgestellten Informationen verwendet, ohne dass die KI zusätzliche Inhalte oder Änderungen einbrachte.
Etwaige zusätzliche Informationen oder Änderungen, die von der KI vorgeschlagen wurden, fanden keine Berücksichtigung.
Diese Vorgehensweise sorgt dafür, dass die Integrität und Authentizität der Arbeit des Autors unbeeinträchtigt bleiben.

Bei der Entwicklung des praktischen Teils dieser Arbeit wurde \emph{ChatGPT} genutzt, um den selbst geschriebenen Code zu optimieren, zu kommentieren, Beispieldaten zu generieren und Fehler zu analysieren.