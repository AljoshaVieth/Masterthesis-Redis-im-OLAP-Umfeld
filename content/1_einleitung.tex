\chapter{Einleitung}
\section{Einsatz von Künstlicher Intelligenz}
Zum Schreiben dieser Arbeit wurde Künstliche Intelligenz (KI) eingesetzt.
Dabei wurden die Tools \enquote{ChatGPT} (GPT-4)~\cite{openai_chatgpt_nodate} der Firma \emph{OpenAI} sowie \enquote{DeepL Write}~\cite{deepl_se_deepl_nodate} von \emph{DeepL SE} verwendet.

\emph{ChatGPT} wurde genutzt, um aus den eigens verfassten Sätzen oder Stichpunkten des Autors ausführlichere Texte zu generieren.
Zusätzlich kam \emph{DeepL Write} zum Einsatz, um die Texte weiter zu verfeinern und ihnen einen akademischen Schreibstil zu verleihen.

KI wurde nicht verwendet, um selbstständig komplette Texte zu formulieren oder neue Informationen zu generieren.
Stattdessen wurden ausschließlich die vom Autor bereitgestellten Informationen verwendet, ohne dass die KI zusätzliche Inhalte oder Änderungen einbrachte.
Etwaige zusätzliche Informationen oder Änderungen, die von der KI vorgeschlagen wurden, fanden keine Berücksichtigung.
Diese Vorgehensweise sorgt dafür, dass die Integrität und Authentizität der Arbeit des Autors unbeeinträchtigt bleiben.

Bei der Entwicklung des praktischen Teils dieser Arbeit wurde \emph{ChatGPT} verwendet, um den selbst geschriebenen Code zu verbessern, Kommentare zu generieren und Fehler zu analysieren.
\section{Hintergrund und Kontext}
\section{Problemstellung}
\section{Forschungsziel}
\section{Gliederung der Arbeit}
\section{Verwandte Arbeit}

% TODO
 TODO: DeepL Write und ChatGPT erwähnen
 