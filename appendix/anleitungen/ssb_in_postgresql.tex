\appendixchapter{SSB in PostgreSQL ausführen}\label{sec:appendix-ssb-in-postgres}

\subsubsection{Docker Container erstellen}
\begin{lstlisting}[caption=Ohne Config, label=code:ohneconfig]
docker run --name postgres-scale-2 -p 5502:5432 -v C:\\Users\\Aljosha\\Documents\\Programming\\HFU\\INM3\\StartSchemaBenchmark\\generated\\scale-2:/tmp -v C:\\Users\\Aljosha\\Documents\\Programming\\HFU\\INM3\\StartSchemaBenchmark\\Postgres\\ssb-postgres:/scripts -e POSTGRES_PASSWORD=pw postgres:latest
\end{lstlisting}

\subsubsection{Gemountete Daten}
\noindent \textbf{scripts:}
\begin{lstlisting}[caption=Scripts im Verzeichnis, label=code:scriptfiles]
explain-analyze.sql  explain.sql  import-date.sql  load.sql  README.md  tables.sql
\end{lstlisting}
Zusätzlich gibt es ein weiteres Skript "get-results.sql" das eine modifizierte Version von "explain-analyze.sql" ist und die Ergebnisse der Queries ausgibt.

\noindent \textbf{tmp:}
\begin{lstlisting}[caption=Temporäre Dateien, label=code:tmpfiles]
customer.tbl  date.tbl  lineorder.tbl  part.tbl  supplier.tbl
\end{lstlisting}

\subsubsection{In Container verbinden}
\begin{lstlisting}[caption=In Container verbinden, label=code:connectcontainer]
docker exec -it postgres-standard sh
\end{lstlisting}

\subsubsection{Postgres shell öffnen}
\begin{lstlisting}[caption=Postgres Shell öffnen, label=code:postgresopen]
psql -U postgres
\end{lstlisting}

\begin{lstlisting}[caption=Verbindungsinformationen, label=code:conninfo]
\conninfo
You are connected to database "postgres" as user "postgres" via socket in "/var/run/postgresql" at port "5432"
\end{lstlisting}

\subsubsection{Datenbank erstellen}
\begin{lstlisting}[caption=Datenbank erstellen, label=code:createdb]
use ssb
\end{lstlisting}

\subsubsection{Tabellen erstellen}
\begin{lstlisting}[caption=Tabellen erstellen, label=code:createtables]
\i /scripts/tables.sql

DROP TABLE
CREATE TABLE
CREATE TABLE
CREATE TABLE
CREATE TABLE
CREATE TABLE
\end{lstlisting}

\begin{lstlisting}[caption=Liste der Tabellen, label=code:listtables]
\dt
           List of relations
 Schema |   Name    | Type  |  Owner
--------+-----------+-------+----------
 public | customer  | table | postgres
 public | date      | table | postgres
 public | lineorder | table | postgres
 public | part      | table | postgres
 public | supplier  | table | postgres
(5 rows)
\end{lstlisting}

\subsubsection{Tabellen mit Daten befüllen}
\begin{lstlisting}[caption=Tabellen mit Daten befüllen, label=code:loaddata]
postgres-# \i /scripts/load.sql

TRUNCATE TABLE
COPY 30000
COPY 2556
COPY 200000
COPY 2000
COPY 6001171
\end{lstlisting}

\subsubsection{Benchmark ausführen}
\begin{lstlisting}[caption=Benchmark ausführen, label=code:runbenchmark]
\i /scripts/explain-analyze.sql
\end{lstlisting}

\subsubsection{Ergebnisse der Queries}
\begin{lstlisting}[caption=Ergebnisse der Queries abrufen, label=code:getresults]
\i /scripts/get-results.sql
\end{lstlisting}

\subsubsection{Quelle der Skripte}
\url{https://github.com/nuko-yokohama/ssb-postgres}~\cite{nukoyokohama_ssb-postgres_2023}
