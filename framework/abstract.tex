\chapter*{Abstract\markboth{Abstract}}
\addcontentsline{toc}{chapter}{Abstract}

\section{Englisch}
Redis is a popular NoSQL database that is frequently used in OLTP applications. Nevertheless, there is limited research on using Redis as stand-alone OLAP software. This thesis analyses whether and how Redis can be used as a stand-alone OLAP application. The data and queries of the Star Schema Benchmark (SSB) are used to compare the performance of Redis and PostgreSQL. Various implementation strategies for the SSB in Redis are analysed and compared. It is shown that Redis can process most SSB queries faster than PostgreSQL with the help of the RediSearch module and denormalised and indexed data. However, Redis requires significantly more memory and data customisation, which can have an impact on costs.

\section{Deutsch}
Redis ist eine populäre NoSQL-Datenbank, die häufig in OLTP-Anwendungen eingesetzt wird. Bislang gibt es jedoch wenig Forschung über den Einsatz von Redis in OLAP-Anwendungen. Diese Arbeit untersucht, ob und wie Redis als eigenständige OLAP-Anwendung genutzt werden kann. Dabei wird mithilfe der Daten und Queries des Star Schema Benchmarks (SSB) ein Vergleich der Leistungsfähigkeit zwischen Redis und PostgreSQL erstellt. Es werden verschiedene Implementierungsstrategien für den SSB in Redis betrachtet und verglichen. Dabei zeigt sich, dass Redis mit Hilfe des Moduls RediSearch und denormalisierten und indizierten Daten die meisten SSB-Anfragen schneller verarbeiten kann als PostgreSQL. Allerdings benötigt Redis deutlich mehr Speicher und eine Anpassung der Daten, was sich auf die Kosten auswirken kann.